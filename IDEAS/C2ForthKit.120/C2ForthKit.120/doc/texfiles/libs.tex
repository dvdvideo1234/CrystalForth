
@chapter Libraries and Calling Forth

@*
Since there is no librarian tool or intermediate binary in the
C2Forth system it is necessary for some other mechanism for
declaring C library functions such as @strong{printf()} etc.
@vskip 0pt
@*
The C2F system allows code output from the C compiler to
reference symbols in the Forth dictionary. This means that the
standard C libraries and user functions can be coded in Forth
and built into the system before the C build.
@vskip 0pt
@*
Supplied with the kit is a minimal implementation of some of the
more useful standard C libs.
@vskip 0pt

@section Coding Library Functions in Forth


@vskip -6pt
@*
As an example, here is the library implementation of the standard
C library function @strong{malloc()}.
@vskip 0pt
@*
First the function prototype for the C compiler is placed in the
header file @emph{stdlib.h}. It has the form:-
@vskip 0pt

@cartouche
@example
   extern void *malloc( size_t s ); @end example
@end cartouche
@*
In a Forth world this can be seen as:-
@vskip 0pt

@cartouche
@example
   : malloc        \ size_t -- void* @end example
@end cartouche
@*
Therefore a suitable definition for @strong{@code{malloc}} in Forth
would be:-
@vskip 0pt

@cartouche
@example
   : _malloc       \ size_t -- void*
     allocate
     if drop 0 then
   ; @end example
@end cartouche
@*
There are three key things to remember about Forth definitions for
C library calls.
@vskip 0pt

@enumerate 1 
@item Each function must have a prototype in a C header file.

@item Names in Forth must have a preceeding underscore character.

@item Input parameters are in reverse order to the C prototype.
@end enumerate

@section Variadic Functions


@vskip -6pt
@*
Implementing a variadic function in Forth is slightly more
complex. With a variadic function you have a fixed formal parameter
list, followed by N other items.
@vskip 0pt
@*
In Forth you know the fixed parameter list, the variadic members
should be pulled off the data stack using PICK as required. Note
that the Forth definition is responsible for keeping the overall
stack effect of the C prototype for FORMAL PARAMETERS ONLY. The
Forth definition should not @strong{@code{DROP}} the variadic members.
@vskip 0pt
@*
Please examine the Forth definition for @strong{_SPRINTF()} in
the @emph{STDIO.FTH} library file.
@vskip 0pt
