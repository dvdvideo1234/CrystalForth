
@chapter VFX Forth Runtime Harness

@*
@emph{C2F_CORE.FTH} provides a reference source for the Forth compiler
support harness required to use C2F on the MPE VFX Forth Kernels.
@vskip 0pt
@*
This harness provides all the directives which occur in the
compiler output source to control forward referencing etc.
@vskip 0pt
@*
Documentation here is in source-code order and should be used as
a roadmap when reading the reference code.
@vskip 0pt

@section Misc Configurations


@vskip -6pt
@vskip 6pt
@findex <periodic-resolve>

@example
variable <periodic-resolve>     \ -- addr @end example

@vskip -2pt
This variable is used to force the system to attempt to resolve
pending forward references each time the PUBLIC directive is
encountered. Performing this extra pass on the resolve (rather than
the default setting of only resolving at the end of a source) will
reduce the amount of heap memory used at compile time, but will
increase compilation time.
@vskip 6pt
@findex periodic-resolve?

@example
: periodic-resolve?     \ -- flag @end example

@vskip -2pt
Returns TRUE if periodic address resolution is required.

@section CPU Specific Tools


@vskip -6pt
@*
These words and utilities are used to generate target code. They
are isolated here since the required settings and actions may
vary from host to host.
@vskip 0pt
@*
These primitives assume an MPE VFX Forth target.
@vskip 0pt
@vskip 6pt
@findex size-of-machine-call

@example
5 constant size-of-machine-call @end example

@vskip -2pt
The size in address-units of a target call instruction. Primarily
used by 'literal'
@vskip 6pt
@findex get-parent-cfa

@example
: get-parent-cfa        \ xt -- addr:32 @end example

@vskip -2pt
When passed the xt of a definition which is a child of a defining
word, this returns the xt of the runtime action's(parent. Used
in the CONSTANT-XT? definition.
@vskip 6pt
@findex nonamegap

@example
: NonameGap     \ loc -- xt @end example

@vskip -2pt
This primitive will return the XT that would be generated if
:NONAME was invoked whilst HERE was at loc. It is used by the
system to record function entry points.
@vskip 6pt
@findex lp@@

@example
code lp@@        \ -- *lp @end example

@vskip -2pt
This definition returns the local frame pointer. The format of the
local frame is defined in the MPE Locals Section of this text.
@vskip 6pt
@findex flush-opt

@example
: flush-opt     \ -- @end example

@vskip -2pt
This directive forces the MPE code optimiser to flush any pending
code trees.
@vskip 6pt
@findex [literal]

@example
: [literal]     \ n -- ; @end example

@vskip -2pt
This directive lays down the target code necessary to generate the
literal value N at runtime.
@vskip 6pt
@findex 'literal'

@example
: 'literal'     \ dp n -- @end example

@vskip -2pt
This directive can change the literal value compiled by [literal]
when HERE was at DP. It is used to resolve forward referenced
data address literals.
@vskip 6pt
@findex constant-xt?

@example
: CONSTANT-XT?          \ xt -- flag @end example

@vskip -2pt
Return TRUE if XT is a child of CONSTANT. Used when CONSTANTS are
defined for addresses.

@section Multitasking and PAUSE


@vskip -6pt
@*
The code in this section can be used with Forth systems that
have a cooperative multitasker called by the word @strong{@code{PAUSE}}.
When the C2F compiler lays a call to a known XT via
@strong{@code{COMPILECALL}} this code is used. If the current value of
@strong{@code{HERE}} is between the variables @strong{@code{COMP-PAUSE-START}}
and @strong{@code{COMP-PAUSE-END}}, a call to the @strong{@code{PAUSE}} is
compiled before the normal call.
@vskip 0pt
@*
This arrangement allows you to specifiy a range of the
dictionary for which target calls will cause a schedule
operation. By default the range is 0 - 0 which turns @strong{@code{PAUSE}}
generation off. The best settings are usually between the end
of the normal Forth kernel and the start of the C compiled
source. This will make calls to the larger library definitions
have a @strong{@code{PAUSE}} but calls to the kernel primitives will not.
@vskip 0pt
@vskip 6pt
@findex set-comp-pause-range

@example
: Set-Comp-Pause-Range  \ start end -- @end example

@vskip -2pt
This definition sets the range of addresses in the dictionary
for which a compiled word will be preceeded by @strong{@code{PAUSE}}.
@vskip 6pt
@findex ?compile-pause

@example
: ?compile-pause        \ where -- @end example

@vskip -2pt
Compiles a @strong{@code{PAUSE}} if the specified address is within the
range set by @strong{@code{SET-COMP-PAUSE-RANGE}}. Note that if there
is no multitasker present, nothing is compiled by this definition.

@section Tools


@vskip -6pt
@vskip 6pt
@findex cksum

@example
: cksum         \ addr len -- cksum @end example

@vskip -2pt
The INET Check sum algorithm. Used to provide a checksum of
generated code to ensure delivery.
@vskip 6pt
@findex icompare

@example
: icompare      \ c-addr1 u1 c-addr2 u2 -- flag @end example

@vskip -2pt
Case insensitive version of the ANS word @strong{@code{COMPARE}}.
@vskip 6pt
@findex upc

@example
: upc           \ v -- 'v @end example

@vskip -2pt
Perform where possible an upper case character conversion.
@vskip 6pt
@findex zstrlen

@example
: zstrlen       \ addr -- len @end example

@vskip -2pt
Given a pointer to a zero-terminated string this will return the
character length excluding the terminator.
@vskip 6pt
@findex zcount

@example
: zcount        \ zaddr -- zaddr len @end example

@vskip -2pt
A zero terminated string version of COUNT.
@vskip 6pt
@findex >pos

@example
: >pos          \ n -- @end example

@vskip -2pt
Used in diagnostic display. Move the cursor X position to column N.
@vskip 6pt
@findex .dword

@example
: .dword        \ n -- @end example

@vskip -2pt
Used in diagnostic display. Show the value N as an unsigned
32 bit hexadecimal.
@vskip 6pt
@findex c@@s

@example
: c@@s           \ addr -- cell @end example

@vskip -2pt
Fetch character from ADDR and sign extend to a CELL.
@vskip 6pt
@findex allot&erase

@example
: allot&erase   \ n -- @end example

@vskip -2pt
Allot N bytes from the dictionary and pre-erase it.
@vskip 6pt
@findex char>cell

@example
: char>cell             \ signed-byte -- signed-int @end example

@vskip -2pt
Sign extend the supplied character to a CELL.

@section Code Stream Handlers


@vskip -6pt
@*
The @strong{codestream} is an area into which strings are built to
@strong{@code{EVALUATE}} in order to perform compilation.
@vskip 0pt
@vskip 6pt
@findex codestream

@example
256 buffer: CodeStream @end example

@vskip -2pt
The temporary buffer into which a string is built.
@vskip 6pt
@findex resetstream

@example
: ResetStream           \ -- @end example

@vskip -2pt
Re-initialises the @strong{@code{CodeStream}} buffer to zero length.
@vskip 6pt
@findex evalstream

@example
: EvalStream            \ -- @end example

@vskip -2pt
Perform an @strong{@code{EVALUATE}} on the code stream.
@vskip 6pt
@findex $>stream

@example
: $>stream              \ c-addr u -- @end example

@vskip -2pt
Append a string to the code stream buffer.
@vskip 6pt
@findex h>stream

@example
: h>stream              \ n -- @end example

@vskip -2pt
Add a string to the end of the code stream buffer which represents
the hexadecimal number N. The number text is preceeded by a $
character as per MPE practice.

@section LABEL Chain Handlers


@vskip -6pt
@*
During compilation of a converted C source file, all labels defined
are kept in a linked list. Each list entry holds the name of the
label, the address of the labels target if known and a set of flags
which describe the label useage. All data pointers and branch
targets have a corresponding label.
@vskip 0pt
@*
Label Struct Format - all entries are 1 cell and in order )
@vskip 0pt


@table @code
@item LBL.link
link, pointer to next label record in list or NULL

@end table



@table @code
@item LBL.*name
Pointer to label name which is a counted string

@end table



@table @code
@item LBL.flags
Flags, describe the label type, see Flags

@end table



@table @code
@item LBL.address
Address, the target address the label equates to

@end table

@*
Label Flags consists of one or more of:
@vskip 0pt


@table @code
@item LF_PUBLIC
Label scope is global.

@end table



@table @code
@item LF_ADDRESSVALID
The Address field of the label structure is valid.

@end table



@table @code
@item LF_DATALABEL
The label represents an internal address. This
 is either a data pointer, or a function local label.

@end table



@table @code
@item LF_CODELABEL
The label represents a function entry point.

@end table



@table @code
@item LF_DELETE
Used internally. The label is marked for deletion

@end table

@vskip 6pt
@findex labelchain

@example
0 Value LabelChain @end example

@vskip -2pt
Either NULL or a pointer to the first label structure.
@vskip 6pt
@findex .label

@example
: .label                \ *LABEL -- @end example

@vskip -2pt
Given a pointer to a label structure this definition will output
a human readable dump of the label information.
@vskip 6pt
@findex .l

@example
: .l                    \ -- @end example

@vskip -2pt
Lists all currently defined labels and their attributes. The
output for each label will be the name, address and flags status as
according to the header displayed on screen.
@vskip 6pt
@findex destroylabelchain

@example
: DestroyLabelChain     \ -- @end example

@vskip -2pt
Destroy the entire current label chain.
@vskip 6pt
@findex addlabel

@example
: AddLabel              \ c-addr u flags address -- @end example

@vskip -2pt
Add a new label entry using the name flags and address supplied.
@vskip 6pt
@findex findlabel

@example
: FindLabel     \ c-addr u -- *LABEL | NULL @end example

@vskip -2pt
Look for a label given a name. Returns either a label structure
pointer or NULL if the label was not found. Please note that under
VFX Forth label names are not case sensitive.
@vskip 6pt
@findex findresolvedlabel

@example
: FindResolvedLabel     \ c-addr u -- c-addr u 0 | address true @end example

@vskip -2pt
Similar to FindLabel except only a label which has a valid address
is accepted. On success the label address and TRUE is returned, if
the label does not exist or has not yet been resolved the original
name and namelen parameters are returned with a FALSE flag.
@vskip 6pt
@findex garbagecollectlbl

@example
: GarbageCollectLBL     \ -- @end example

@vskip -2pt
Walks the label list removing and deallocating any structures
marked for delete.
@vskip 6pt
@findex exportlbl

@example
: ExportLBL             \ *lbl -- @end example

@vskip -2pt
Exports a label from a label chain structure supplied into the
current Forth definitions wordlist. Code labels are built by
generating a : definition of the same name which will pass the
compiled codes address to @strong{@code{EXECUTE}}. Non-Code labels are
built as @strong{@code{CONSTANT}}s.
@vskip 6pt
@findex patchlabel

@example
: PatchLabel            \ address iflag *LABEL -- @end example

@vskip -2pt
Modify the supplied label structure to use the supplied address
and combine the supplied flags with the existing ones. This is
used to resolve a label which had no valid address when defined.
A fatal error has occured if any attempt is made to patch a label
which already has a valid target address.
@vskip 6pt
@findex newlabel

@example
: NewLabel              \ address iflag c-addr u -- @end example

@vskip -2pt
This definition is similar to @strong{@code{AddLabel}} except the @strong{@code{LF_ADDRESSVALID}}
flag is automatically added. Used as a shortcut for those times
when a known address label is to be added.
@vskip 6pt
@findex (label)

@example
: (label)               \ address iflag "name" -- @end example

@vskip -2pt
The internal factor used to create most labels. A label name is
parsed from the input buffer and if it exists the address and @emph{iflag}
are passed to @strong{@code{PatchLabel}}. If the label does not
exist it is created by @strong{@code{NewLabel}}.
@vskip 6pt
@findex removelocallabels

@example
: RemoveLocalLabels     \ -- @end example

@vskip -2pt
Destroys the local label list by marking each entry for deletion
and then calling calling @strong{@code{GarbageCollectLBL}}.
@vskip 6pt
@findex exportpublics

@example
: ExportPublics         \ -- @end example

@vskip -2pt
Exports all public labels to the Forth namespace. Works by calling
@strong{@code{EXPORTLBL}} for each local label entry with the public attribute and
then marking it for delete.

@section Forward Reference Chain Handlers


@vskip -6pt
@*
Address literals and branch targets are referenced by symbol name
during compilation. Since C generated code may forward reference
such symbols a chain of unresolved targets is kept during the "C"
build. Each entry in the chain holds the symbol name, a numeric
offset to apply to the symbol, the dictionary pointer for the
target code to patch and the type of patch to apply.
@vskip 0pt
@vskip 6pt
@findex ff_cell

@example
$00000001 constant FF_CELL      \ -- x @end example

@vskip -2pt
This flag when set in the FW.flags field indicates the target code
to be patched is a CELL, otherwise the code to be patched is the
literal code as generated by [literal].
@*
The FW(Forward reference struct has the following fields:
@vskip 0pt


@table @code
@item FW.link
link, pointer to next record in list or NULL

@end table



@table @code
@item FW.*name
Pointer to label name which is a counted string. A zero
 in this field indicates invalid FW and the records is
 removed on the next garbage collect.

@end table



@table @code
@item FW.here
Target dictionary location to perform patch operation.

@end table



@table @code
@item FW.offset
Offset to apply to address of symbol in *name before
 patching dictionary.

@end table



@table @code
@item FW.flags
Flags, describe the patch type. Either 0 or FF_CELL.

@end table

@vskip 6pt
@findex fwchain

@example
0 Value FWChain \ -- addr|0 @end example

@vskip -2pt
Either NULL or pointer to the first forward reference record.
@vskip 6pt
@findex .fw

@example
: .fw                   \ *FW -- @end example

@vskip -2pt
Given a pointer to a FW structure display it in human readable
form.
@vskip 6pt
@findex .f

@example
: .f                    \ -- @end example

@vskip -2pt
Display list of all pending forward references, their target
location and type of patch to apply.
@vskip 6pt
@findex destroyfwchain

@example
: DestroyFWChain        \ -- @end example

@vskip -2pt
Destroy the entire forward reference chain.
@vskip 6pt
@findex $addlitfw

@example
: $AddLitFW             \ c-addr u -- *FW @end example

@vskip -2pt
Add a new forward reference record for symbol whose name is
passed as @emph{C-ADDR U}. Returns a pointer to the record to
allow the addresses and flags to be set by the caller.
@vskip 6pt
@findex resolve

@example
: Resolve               \ *FW -- res? @end example

@vskip -2pt
If possible resolve the forward reference whose structure is at
@emph{*FW}. Returns TRUE if the reference was found or false if the
name does not exist or the address has not been resolved yet.
@vskip 6pt
@findex garbagecollectfw

@example
: GarbageCollectFW      \ -- @end example

@vskip -2pt
Destroy any forward reference records with the FW.*name field
at zero.
@vskip 6pt
@findex resolvefw

@example
: ResolveFW             \ -- @end example

@vskip -2pt
Attempt to resolve all forward references and delete any which
have succesfully been resolved.

@section Compilers


@vskip -6pt
@*
Provides the harness definitions for the compilation of address
literals and branches. Each compilation directive will attempt
to resolve the required address and generate optimial code, for
an unresolved address and forward reference is generated and some
less optimal but patchable code is generated.
@vskip 0pt
@vskip 6pt
@findex name-store

@example
#256 buffer: name-store \ -- addr @end example

@vskip -2pt
A buffer to store the first token after a directive, i.e. the target
label name. Performed since VFX parses text at @strong{@code{HERE}} which will
be corrupted during code generation.
@vskip 6pt
@findex >name-store

@example
: >name-store           \ c-addr u -- 'c-addr u @end example

@vskip -2pt
Copy the string @emph{C-ADDR U} to the name-store buffer for use later.
@vskip 6pt
@findex parsenumericoffset

@example
: ParseNumericOffset    \ "offset" -- val @end example

@vskip -2pt
Parse the next token from the input stream and attempt to convert
to a signed numeric offset. @strong{@code{ABORT}}s if the next token is not
numeric. This is used to parse and set the address offset as
specified by directives such as @strong{@code{COMMAADDRLIT}}.
@vskip 6pt
@findex commaaddr

@example
: CommaAddr             \ "symbol" "offset" -- @end example

@vskip -2pt
Generate code to place the address of "symbol" with "offset" into
the dictionary as a CELL via comma. If the address is already
known it is simply passed to comma, otherwise a 0 value is compiled
and a FF_CELL forward reference added to the FW chain.
@vskip 6pt
@findex compileaddrlit

@example
: CompileAddrLit        \ "symbol" "offset" -- @end example

@vskip -2pt
Similar to @strong{@code{CommaAddr}} except rather than laying the data as a CELL
in the dictionary the runtime code for a literal is compiled.
As with @strong{@code{CommaAddr}} if the label is resolved the literal is used
directly (as though it was present in the source) otherwise a
patchable literal is compiled via @strong{@code{[LITERAL]}} and a forward reference
is added.
@vskip 6pt
@findex compilecall

@example
: CompileCall           \ "symbol" -- @end example

@vskip -2pt
Compile a call to the address named "symbol". If the address is
known it is passed to @strong{@code{COMPILE,}} otherwise  "symbol" is treated as
with @strong{@code{COMPILEADDRLIT}} and a call to the Forth @strong{@code{EXECUTE}} is compiled
afterwards. Also note that when compiling a resolved symbol the
target address is passed to @strong{@code{?COMPILE-PAUSE}} as documented above in
the section on multitasking.
@vskip 6pt
@findex compilejump

@example
: CompileJump           \ -- @end example

@vskip -2pt
Compile code into target to perform a JUMP to the address in TOS.
@vskip 6pt
@findex compilebranch

@example
: CompileBranch         \ "symbol" -- @end example

@vskip -2pt
Compile a jump the address reference by "symbol". This is acheived
by compiling a literal as with CompileAddrLit and then calling
@strong{@code{COMPILEJUMP}} to lay the branch code.
@vskip 6pt
@findex compilecondbranch

@example
: CompileCondBranch     \ "symbol" -- @end example

@vskip -2pt
Lays code for a conditional variant of @strong{@code{COMPILEBRANCH}}. The normal
method of achieving this is to postpone IF COMPILEBRANCH THEN.

@section Label Definitions


@vskip -6pt
@*
These definitions form part of the language extensions as used by
the output from the c2forth translator. They each record a specific
kind of label.
@vskip 0pt
@vskip 6pt
@findex flabel

@example
: FLabel                \ "name" -- @end example

@vskip -2pt
Adds a new code-entry point label. The actual address represented
is the target XT address which will be returned if @strong{@code{:NONAME}} is
invoked at the current value of @strong{@code{HERE}}. If the label already exists
it has the @strong{@code{LF_CODELABEL}} flag added to its current incarnation.
@vskip 6pt
@findex label

@example
: Label                 \ "name" -- @end example

@vskip -2pt
Adds a new non-entry point label. These labels are either reference
points within a definition, e.g. branch targets, or are pointers to
data items. As with @strong{@code{FLabel}}, an existing label is patched. This time
with the @strong{@code{LF_DATALABEL}} flag.
@vskip 6pt
@findex public

@example
: Public                \ "name" -- @end example

@vskip -2pt
Add a new label of type @strong{@code{LF_PUBLIC}}. This operation allways adds a
new label, and does not specify a type. It will usually be immediately
followed by a @strong{@code{FLABEL}} or @strong{@code{LABEL}} directive to specify type.
@vskip 6pt
@findex extern

@example
: Extern                \ "name" -- @end example

@vskip -2pt
@strong{@code{EXTERN <label>}} will eventually be used to mark external labels.
Currently C2F ignores these directives and everything is treated as
global.

@section Variadic Function Support


@vskip -6pt
@*
The calling method for variadic functions involves placing the
known stack depth into the first local variable, building a frame
with the specified parameters and using an equivalent to Forth's
@strong{@code{PICK}} instruction to grab the variadic parameters. These three
definitions are used at runtime to support variadics.
@vskip 0pt
@vskip 6pt
@findex _vm_va_depth

@example
: _vm_va_depth          \ -- i @end example

@vskip -2pt
Return the contents of the first local variable.)
@vskip 6pt
@findex _vm_depth

@example
: _VM_depth             \ -- n @end example

@vskip -2pt
Return data stack depth. A sanity check, how far down the data
stack can a variadic function safely perform @strong{@code{PICK}}.)
@vskip 6pt
@findex _vm_pick

@example
: _VM_pick              \ n -- c @end example

@vskip -2pt
An alias for @strong{@code{PICK}} to complete the isolation layer.

@section Outer Harness


@vskip -6pt
@*
The definitions used to control the compilation of C files within
a Forth interpreter.
@vskip 0pt
@vskip 6pt
@findex c-build-start-dp

@example
variable        c-build-start-dp @end example

@vskip -2pt
A variable which records the value of HERE when C compile mode
is enabled. Simply used to report the size of generated target code
on completion.
@vskip 6pt
@findex [eof]

@example
: [EOF]                 \ -- @end example

@vskip -2pt
This definition is invoked at the end of each individual source
include during a C build. It resolves forward references where
possible, deletes the source-file local labels and exports public
symbols to the Forth namespace.
@vskip 6pt
@findex "c"

@example
: "C"                   \ -- @end example

@vskip -2pt
Begin "C" source code build. Records start HERE position and ensures
the local label and forward reference chain are clear.
@vskip 6pt
@findex "forth"

@example
: "FORTH"               \ -- @end example

@vskip -2pt
End a "C" source code build. Reports unresolved forward references
and size/checksum of generated code.
@vskip 6pt
@findex #included

@example
: #included             \ c-addr u -- @end example

@vskip -2pt
A version of the ANS word @strong{@code{INCLUDED}} which automatically
calls @strong{@code{[EOF]}} on completion. Used to include each
individual C file between the @strong{@code{"C"}} and @strong{@code{"FORTH"}}
directives.
@vskip 6pt
@findex #include

@example
: #include              \ "name" -- @end example

@vskip -2pt
A version of the ANS INCLUDE word. Performs as #INCLUDED.
