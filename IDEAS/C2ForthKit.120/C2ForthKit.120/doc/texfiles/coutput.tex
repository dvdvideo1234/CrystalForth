
@chapter Forth Style Output from Compiler

@*
This section describes the format of the compiler output which is
to be built on a Forth System. In many ways the output is Forth
but with certain necessary extensions. The local variable format
may vary between versions of the C compiler and Forth target.
@vskip 0pt

@section Simple C program


@vskip -6pt
@*

@vskip 0pt

@cartouche
@example
void main( void )
@{
        int a;
        a = 0;
        while( a < 26 )
        @{
                putchar( a+65 );
                a++;
        @}
        printf(("\n"));
@} @end example
@end cartouche
@vskip 6.0in

@section Output from the C to Forth Compiler


@vskip -6pt

@cartouche
@example
PUBLIC _main
FLABEL _main
:NONAME

        MAKELVS [ #0 w, #1 w, ]
        #0

        [ #-4 lv, ]
        !
        CompileBranch @@3
LABEL  @@2

        [ #-4 lv, ]
        @@
        #65
        +
        CompileCall _putchar
        drop
        #1

        [ #-4 lv, ]
        +!
LABEL  @@3

        [ #-4 lv, ]
        @@
        #26
        <
        CompileCondBranch @@2
        CompileAddrLit @@5 #0
        CompileCall _printf
        drop
LABEL  @@1

        RELLVS
; drop

EXTERN _printf
EXTERN _putchar
LABEL  @@5
                $0A c,
                $00 c, @end example
@end cartouche

@section Binary Generated on VFX Forth


@vskip -6pt

@cartouche
@example
Target    Assembler/Binary Data Compiled               Correlation To
Address                                                Original C Source


00469ED6  call    MAKELVS                              void main( void )
00469EDB    Inline data16: 0000                        @{
00469EDD    Inline data16: 0001                          int a;

00469EDF  mov     dword Ptr [edi+-04], 00000000          a = 0;

00469EE6  call    LIT                                    while ( a<26 )
00469EEB    Inline data32: 00469F14 ( Label @@3 )
00469EEF  push    ebx
00469EF0  mov     ebx, [ebp]
00469EF3  lea     ebp, [ebp+04]
00469EF6  ret

Label @@2  mov     edx, [Edi+-04]                         putchar( a+65 )
00469EFA  add     edx, 41
00469EFD  lea     ebp, [ebp+-04]
00469F00  mov     [ebp], ebx
00469F03  mov     ebx, edx
00469F05  call    _PUTCHAR
00469F0A  add     [edi+-04], 01                          a++;
00469F0E  mov     ebx, [ebp]
00469F11  lea     ebp, [edp+04]

Label @@3  cmp     [edi+-04], 1a                          .... Test code for
00469F18  jnl/ge  00469F24                               .... WHILE clause
00469F1E  push    00469EF7 ( Label @@2 )
00469F23  ret

00469F24  call    LIT                                    printf("\n");
00469F29    Inline data32: 00469F3E ( Label @@5 )
00469F2D  call    _PRINTF
00469F32  mov     ebx, [ebp]
00469F35  lea     ebp, [ebp+04]

00469F38  call    RELLVS                               @}
00469F3D  ret

Label @@5    Inline data8: 0                            .... String Literal
00469F3F    Inline data8: 0 @end example
@end cartouche
@*
The output is largely Forth code with some non-standard
words being used. These "strange" definitions are Forth compiler
directives built in a harness file on top of the Forth Kernel by
exploiting Forth's ease of extensibility.
@vskip 0pt
@*
The compiler aids are split into 4 groups: Label handling, Control
flow, data address literals and local frame access. Both the flow
control directives and the data address literal directives support
forward referencing by name and offset.
@vskip 0pt
@*
Most "traditional" C compilers produce code from top to bottom and
assume a back-end assembler which can do 2 pass compilation. The
Forth system cannot easily be coerced into a dual pass compiler
so any attempt to reference an unresolved label adds a record into
a forward reference chain, and generates a form of binary whereby
the address can be patched up later. For this reason there are
"magic" compiler directives for anything which may be formed by
using a label which could be forward references.
@vskip 0pt

@section Code Generation Extensions


@vskip -6pt
@*
The following definitions are used to provide required compilation
behaviour not standard in Forth. All definitions for data addresses
and flow control are @strong{@code{IMMEDIATE}}.
@vskip 0pt
@*
For details of how to implement these directives as well as formal
stack comments and operation descriptions, please see the section
describing the current reference harness.
@vskip 0pt

@subsection Label Creation


@vskip -4pt
@*
There are four compiler directives which create and modify labels.
All branch targets and data addresses have a label.
@vskip 0pt


@table @code
@item PUBLIC
Parses a <name> and creates a label entry with no
 specified type or address.

@end table



@table @code
@item EXTERN
Currently not used since C2F assumes all labels are
 global and anything not found in the current C source
 will be searched for in the Forth dictionary.

@end table



@table @code
@item FLABEL
Used to define an address for a function entry point.
 FLABEL will either patch the address and type of an
 already defined PUBLIC label (creating a public function
 label), or will create a new code label for the current
 address. Note that the address stored by FLABEL is not
 the current value of HERE, but the value HERE will be at
 after executing a :NONAME. IE, the label address stored
 will be the XT of the following headerless code.

@end table



@table @code
@item LABEL
Behaves as FLABEL for internal branch targets and
 data addresses. The value stored will be the value of
 HERE. Therefore LABEL cannot be used for function entry
 points.

@end table


@subsection Local Frame Access


@vskip -4pt
@*
Local frame access is performed by exploiting the implementation
of local variables which is common to all MPE targets. For the
purposes of the C2F compiler we need three definitions:
@vskip 0pt


@table @code
@item MAKELVS
This definition builds a local frame of a specified size
 and moves any parameters from the data stack into the
 local frame. The frame itself is usually created by
 "dropping" the return stack pointer.
 MAKELVS is always followed by a 32 bit number where the
 high 16 bits represent the number of arguments and the
 low 16 bits represent the size in bytes of the local
 data space.

@end table



@table @code
@item RELLVS
This definition will release the last allocated locals
 frame and restore the return stack pointer.

@end table



@table @code
@item LV,
An internal definition which when executed will take
 a literal frame offset value and compile the code
 necessary to put the address of the local frame + offset
 onto the data stack at runtime.

@end table

@*
The local variable format used was designed during the SENDIT
and PRACTICAL projects. The frame is built on the return stack.
@vskip 0pt

@cartouche
@example
argn
...        paramteters
arg1
----
previous return stack pointer (RSP)
previous locals pointer (LP)
---- new LP points to old LP
uninitialised locals
---- new RSP points here @end example
@end cartouche
@*
In an environment with interrupts, the code laid or performed
by @strong{@code{RELLVS}} must restore LP before RSP.
@vskip 0pt

@subsection Data Address Literals


@vskip -4pt
@*
There are two definitions which are used whenever a data address
literal reference needs to be compiled.
@vskip 0pt


@table @code
@item CompileAddrLit
Compiles code which will place the address of
 the label specified on the top of the data stack
 at runtime.

@end table



@table @code
@item CommaAddrLit
Places the address of the specified label into
 the next cell in the dictionary.

@end table

@*
Both these directives have the same argument syntax. The first
token is the label name, and the second token is a signed 32bit
offset literal. Forward referencing is built into these two
compilers such that, if the label has allready been resolved the
address is used directly either as a literal or via @strong{@code{,}}
(comma). If the label has not yet been defined or resolved
then code generation is completed assuming a value of 0 and
a forward reference entry is generated for later resolution.
Any generated code which requires a forward reference should
be built in such a way that it can have its value changed
later. This is implementation specific.
@vskip 0pt

@subsection Control Flow


@vskip -4pt
@*
There are three definitions which are used to perform branches.
@vskip 0pt


@table @code
@item CompileCall
Lay code to CALL the address represented by
 the given label. If the address is allready
 known, this is a COMPILE, otherwise some code is
 compiled whereby the XT is built into the
 dictionary as a CELL which can be updated when
 the address for the label has been resolved.

@end table



@table @code
@item CompileBranch
Similar to CompileCall except it behaves as an
 assembler JUMP instruction. There is no direct
 Forth equivilant to this operation, on a number
 of systems it can be acheived by performing
 calling a definition which performs >R. This is
 very implementation specific but common.

@end table



@table @code
@item CompileCondBranch
Compiles code to perform a branch if
 top-of-stack is non-zero. Some systems may have
 the word ?BRAN for this operation, other more
 ANS compliant targets can define this as using
 POSTPONE IF POSTPONE COMPILEBRANCH POSTPONE THEN

@end table

