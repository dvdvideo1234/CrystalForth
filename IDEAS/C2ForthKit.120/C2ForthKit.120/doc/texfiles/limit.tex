
@chapter Limitations

@*
You would have to go a long way to find two langauges so far apart
as C and Forth so obviously this system has some limitations.
@vskip 0pt

@section Short Data Types


@vskip -6pt
@*
Some C programmers are used to the data type SHORT on a 32bit
system being 16 bits. The current C compiler tool treats all SHORTs
as INTs.
@vskip 0pt

@section Floating Point


@vskip -6pt
@*
There is absolutly no support in this system for floating point
numbers at all.
@vskip 0pt

@section Namespace Visibility and Linking


@vskip -6pt
@*
At the moment the runtime code on the Forth kernel makes no
attempt to obey the C EXTERN directive. All public labels are
built in the global name space, i.e. Forth's current)vocabulary.
Also since the delivery mechanism for this system is source
code to an open Forth, there is no binary library format to
link against.
@vskip 0pt

@section Standard C Library


@vskip -6pt
@*
Due to the lack of a binary library format there is no
method of compiling C libraries from C source code and
storing the object code for linking. C2Forth has been
designed to link C to Forth; since the Forth search-order
is used to resolve symbols during compilation it is possible
to write library functions in Forth(with a preceeding
underscore in the name) and call that definition from the C
code by name. (You will need a header file and prototype for
the C compiler.)C2Forth contains a very limited subset of
the C standard library as Forth source code along with C
header files.
@vskip 0pt

@section Target Forth Kernel


@vskip -6pt
@*
This system has been designed to exploit the optimising Forth
kernels built by MPE. As such some tradeoffs have been made
between portability of output code and speed.
@vskip 0pt
@*
The VFX kernels from MPE all have the following characteristics:
@vskip 0pt

@itemize @bullet 
@item 32 bit CELL, 8 bit CHAR.

@item Native Code Generation.

@item An XT (or CFA) is an executable address not some other id.

@item Return Stack access is permitted outside of the current definition.

@item Headerless definitions supported.

@item MPE standard local variable code present.
@end itemize
@*
There is no technical reason why the required kernel support
cannot be built on any Forth System. The next section describes
the output form of the Forth source and the functionality
required above and beyond ANS Forth. The harness/kernel file
supplied is for our VFX targets and as such, the nearer a
Forth Kernel is to our specification, the easier it will be
to port the existing harness.
@vskip 0pt
