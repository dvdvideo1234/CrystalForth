
@chapter C library

@*
C2Forth provides a minimal version of the C standard library.
The minimal version is enough for most embedded applications.
You can extend the the library yourself in Forth or C.
@vskip 0pt

@section ASSERT - Runtime Debugging Exception


@vskip -6pt
@*
The @strong{ASSERT()} mechanism provides runtime error checking
facilities. The parameter to @strong{ASSERT()} is a small piece
of RT code which when evaluated to 0 at runtime causes the
program to halt and display the original source file and
line number.
@vskip 0pt
@example
void assert( <expression> ) @end example
@*
If <expression> evaluates to 0 (ZERO)at runtime, cause a halt.
@vskip 0pt
@vskip 6pt
@findex ___assert

@example
: ___assert     \ failedexpr line file -- @end example

@vskip -2pt
Runtime for a failed @strong{assert()}.

@section CONIO - Extended Console IO functions


@vskip -6pt
@*
The conio library holds functions relating to console input
and output which, whilst useful, are not part of the ANSI/ISO
library standards.
@vskip 0pt
@example
int kbhit( void ) @end example
@*
Returns non-zero is a key has been pressed, otherwise
returns 0. This is a non-blocking "peek" operation on
stdin.
@vskip 0pt
@vskip 6pt
@findex _kbhit

@example
: _kbhit        \ -- flag @end example

@vskip -2pt
Return true when a key has been pressed. Equivalent to the
Forth word @strong{@code{KEY?}}.

@section CTYPE - Character Type Query and conversion


@vskip -6pt
@*
This is an implementation of the ANSI "C" library for testing
and converting individual characters in the standard character
form.
@vskip 0pt

@subsection C prototypes


@vskip -4pt
@example
char isalpha( char t ); @end example
@*
Returns non-zero if the character t is either an uppercase or
lowercase alphabetic character. ('A'..'Z' or 'a'..'z' ).
@vskip 0pt
@example
char isascii( char t ); @end example
@*
Returns non-zero if the character t is within the 7 bit
ASCII alphabet.
@vskip 0pt
@example
char isblank( char t ); @end example
@*
Is the supplied character a blank?
@vskip 0pt
@example
char iscntrl( char t ); @end example
@*
Is the supplied character a control character?
@vskip 0pt
@example
char isdigit( char t ); @end example
@*
Is the supplied character a numeric digit?
@vskip 0pt
@example
char isxdigit( char t ); @end example
@*
Is the supplied character a hexadecimal digit?
@vskip 0pt
@example
char isgraph( char t ); @end example
@*
Is the supplied character a graphics character?
@vskip 0pt
@example
char islower( char t ); @end example
@*
Is the supplied character a lower case alphabetic?
@vskip 0pt
@example
char isupper( char t ); @end example
@*
Is the supplied character an uppercase alphabetic?
@vskip 0pt
@example
char isspace( char t ); @end example
@*
Is the supplied character a space?
@vskip 0pt
@example
char ispunct( char t ); @end example
@*
Is the supplied character a puncuation mark?
@vskip 0pt
@example
char isprint( char t ); @end example
@*
Is the supplied character printable?
@vskip 0pt
@example
char isalnum( char t ); @end example
@*
Is the supplied character an alphanumeric?
@vskip 0pt
@example
char tolower( char t ); @end example
@*
Convert char to lower case where possible.
@vskip 0pt
@example
char toupper( char t ); @end example
@*
Convert char to upper case where possible.
@vskip 0pt

@subsection Forth implementation


@vskip -4pt
@*
Conversion and testing is controlled by a type table.
@vskip 0pt

@example
: _U    $01 c, ; immediate          \  UpperCase
: _L    $02 c, ; immediate          \  LowerCase
: _N    $04 c, ; immediate          \  Numeric
: _S    $08 c, ; immediate          \  WhiteSpace
: _P    $10 c, ; immediate          \  Punctuation
: _C    $20 c, ; immediate          \  Control
: _X    $40 c, ; immediate          \  Xdigit
: _B    $80 c, ; immediate          \  Space
: _CS   $28 c, ; immediate          \  Control or WhiteSpace
: _SB   $88 c, ; immediate          \  Space character
: _UX   $41 c, ; immediate          \  Uppercase or XDigit
: _LX   $42 c, ; immediate          \  Lowercase or XDigit
 @end example

@vskip 6pt
@findex ctype-table

@example
create ctype-table      \ -- addr @end example

@vskip -2pt
Type table for ASCII characters.
@vskip 6pt
@findex [ctype]

@example
: [ctype]       \ char -- val @end example

@vskip -2pt
Return the type value of a character.
@vskip 6pt
@findex _isalpha

@example
: _isalpha      \ char -- flag @end example

@vskip -2pt
Returns non-zero if the character t is either an uppercase or
lowercase alphabetic character. ('A'..'Z' or 'a'..'z' ).
@vskip 6pt
@findex _isascii

@example
: _isascii      \ char -- flag @end example

@vskip -2pt
Returns non-zero if the character is within the 7 bit
ASCII alphabet.
@vskip 6pt
@findex _isblank

@example
: _isblank      \ char -- flag @end example

@vskip -2pt
Is the supplied character a blank?
@vskip 6pt
@findex _iscntrl

@example
: _iscntrl      \ char -- flag @end example

@vskip -2pt
Is the supplied character a control character?
@vskip 6pt
@findex _isdigit

@example
: _isdigit      \ char -- flag @end example

@vskip -2pt
Is the supplied character a numeric digit?
@vskip 6pt
@findex _isxdigit

@example
: _isxdigit     \ char -- flag @end example

@vskip -2pt
Is the supplied character a hexadecimal digit?
@vskip 6pt
@findex _isgraph

@example
: _isgraph      \ char -- flag @end example

@vskip -2pt
Is the supplied character a graphics character?
@vskip 6pt
@findex _islower

@example
: _islower      \ char -- flag @end example

@vskip -2pt
Is the supplied character a lower case alphabetic?
@vskip 6pt
@findex _isupper

@example
: _isupper      \ char -- flag @end example

@vskip -2pt
Is the supplied character an uppercase alphabetic?
@vskip 6pt
@findex _isspace

@example
: _isspace      \ char -- flag @end example

@vskip -2pt
Is the supplied character a space?
@vskip 6pt
@findex _ispunct

@example
: _ispunct      \ char -- flag @end example

@vskip -2pt
Is the supplied character a puncuation mark?
@vskip 6pt
@findex _isprint

@example
: _isprint      \ char -- flag @end example

@vskip -2pt
Is the supplied character printable?
@vskip 6pt
@findex _isalnum

@example
: _isalnum      \ char -- flag @end example

@vskip -2pt
Is the supplied character an alphanumeric?
@vskip 6pt
@findex _tolower

@example
: _tolower      \ char -- 'char @end example

@vskip -2pt
Convert char to lower case where possible.
@vskip 6pt
@findex _toupper

@example
: _toupper      \ char -- 'char @end example

@vskip -2pt
Convert char to upper case where possible.

@section STDIO - Standard IO Model


@vskip -6pt
@*
The standard IO model has been stripped for C2Forth. There is no
file support, and the STDIO system relates to the debug serial
port on the target. These functions are for DEBUG only and should
not be used for runtime code since they affect performance. At
runtime communication should be between the target and the host
rather than the debug-console.
@vskip 0pt
@vskip 6pt
@findex _stdout

@example
0 constant _stdout      \ -- x @end example

@vskip -2pt
The C @strong{stdout} value. Change as required by your host.
@vskip 6pt
@findex _stdin

@example
1 constant _stdin       \ -- x @end example

@vskip -2pt
The C @strong{stdin} value. Change as required by your host.
@vskip 6pt
@findex _stderr

@example
2 constant _stderr      \ -- x @end example

@vskip -2pt
The C @strong{stderr} value. Change as required by your host.
@vskip 6pt
@findex _puts

@example
: _puts         \ char* -- len @end example

@vskip -2pt
Write a string of text to the the debug output channel
@example
  int puts( const char *text ); @end example
@vskip 6pt
@findex _sprintf

@example
: _sprintf      \ .... fmt buff -- .... len @end example

@vskip -2pt
Write a string of text using the printf() syntax to a memory
array.
@example
  int sprintf( char *b, const char *fmt, ... ); @end example
@vskip 6pt
@findex _printf

@example
: _printf      \ .... fmt -- .... len @end example

@vskip -2pt
Write a formated string to the debug console.
@example
  int printf( const char *fmt, ... ); @end example
@vskip 6pt
@findex _getchar

@example
: _getchar      \ -- char @end example

@vskip -2pt
Wait for and return the next key pressed.
@example
  int getchar( void ); @end example
@vskip 6pt
@findex _putchar

@example
: _putchar      \ char -- char @end example

@vskip -2pt
Transmit the character.
@vskip 6pt
@findex _sscanf

@example
: _sscanf        \ ????? @end example

@vskip -2pt
Not implemented.

@section STDLIB - Standard Library


@vskip -6pt
@vskip 6pt
@findex #exitchain

@example
#32 constant #ExitChain                 \ -- u @end example

@vskip -2pt
Maximum number of entries in the exit chain.
@vskip 6pt
@findex cells

@example
#ExitChain cells buffer: ExitChain      \ -- addr @end example

@vskip -2pt
Buffer for exit chain entries.
@vskip 6pt
@findex exitchainidx

@example
0 Value ExitChainIdx                    \ -- 0|addr @end example

@vskip -2pt
Anchor for exit chain.
@vskip 6pt
@findex _abort

@example
: _abort        \ -- @end example

@vskip -2pt
Terminate the current program no matter what.
@example
  void abort( void ); @end example
@vskip 6pt
@findex _exit

@example
: _exit         \ n -- @end example

@vskip -2pt
As with @strong{abort()} except with a return status code and the
fact that an exit() is not considered abnormal termination.
@example
  void exit( int status ); @end example
@vskip 6pt
@findex _atexit

@example
: _atexit       \ xt -- ior @end example

@vskip -2pt
Specify a function which will be executed when @strong{exit()} occurs.
@example
  int atexit( void(*func)(void)); @end example
@vskip 6pt
@findex _malloc

@example
: _malloc       \ size -- ptr | 0 @end example

@vskip -2pt
Attempt to allocate S bytes of heap and return a pointer.
@example
  void *malloc( size_t s ); @end example
@vskip 6pt
@findex _free

@example
: _free         \ ptr -- @end example

@vskip -2pt
Release @strong{malloc()}ed memory.
@example
  void free( void *ptr ); @end example
@vskip 6pt
@findex _realloc

@example
: _realloc      \ size ptr -- 'ptr | 0 @end example

@vskip -2pt
Attempt to resize the allocated block P to the new size S.
@example
  void *realloc( void *p, size_t s ); @end example
@vskip 6pt
@findex _atoi

@example
: _atoi         \ s -- n @end example

@vskip -2pt
Convert an ascii string to a decimal number.
@example
  int atoi( const char *s ); @end example

@section STRING - Simple String Tools


@vskip -6pt
@*
As with all the other libraries, this one has been stripped.
Sufficient functionality is provided for most programs and
those functions missing can be easily written in C using the
ones supplied.
@vskip 0pt
@vskip 6pt
@findex _memset

@example
: _memset       \ size value source -- source @end example

@vskip -2pt
Set memory (bytes).
@example
  void *memset( void *, int c, size_t s ); @end example
@vskip 6pt
@findex _strlen

@example
: _strlen       \ z$ -- len @end example

@vskip -2pt
Return the length of the supplied string not including the
zero terminator.
@example
  size_t strlen( const char *s ); @end example
@vskip 6pt
@findex _strcpy

@example
: _strcpy       \ source dest -- dest @end example

@vskip -2pt
Copy the string S to the char array at D.
Returns the pointer to the destination.
@example
  char *strcpy( char *d, const char *s ); @end example
@vskip 6pt
@findex _memcpy

@example
: _memcpy       \ len source dest -- dest @end example

@vskip -2pt
Copy memory.
@example
void *memcpy( void *d, const void *s, size_t len ); @end example
@vskip 6pt
@findex _strtol

@example
: _strtol       \ base **endptr *s -- val @end example

@vskip -2pt
Attempt to convert the string S to a long integer using
BASE as the radix. ENDPTR points to a CHAR* pointer which
will hold the address of the first part of the source string
which cannot be converted as a numeric digit.
@example
  long int strtol( const char *s, char **endptr, int base ); @end example
@vskip 6pt
@findex _memcmp

@example
: _memcmp       \ len s1 s2 -- v @end example

@vskip -2pt
Compare two blocks of memory.
@example
int memcmp( const void *s1, const void *s2, size_t n ); @end example
@vskip 6pt
@findex _strcat

@example
: _strcat       \ source dest -- dest @end example

@vskip -2pt
Append string S2 onto the char array at S1.
@example
  char *strcat( char *s1, const char *s2 ); @end example
