
@chapter Usage with VFX Forth

@*
This section explains how to compile the supplied test programs
and build them on the ProForth VFX system using the reference
harness.
@vskip 0pt

@section Building the Test Apps


@vskip -6pt
@*
In the distributions \TESTS directory there are three sample
applications.
@vskip 0pt


@table @code
@item BANNER
Does large screen pretty printing, similar to the Unix
 banner program.

@end table



@table @code
@item TESTHEAP
Performs a stress test on the system heap.

@end table



@table @code
@item VARTEST
Test program for simply variadic functions.

@end table

@*
By entering MAKE <testname> in the \TESTS directory you can build
each test program. The translated source for each file will have
a CXX file extension. This is the Forth style output.
@vskip 0pt
@*
MAKE itself is a batch file which calls OMAKE (in the \BIN folder)
and uses the supplied .MAK file which can be used as a template
for all builds.
@vskip 0pt

@section Running the Test Apps


@vskip -6pt
@*
In order to compile and run the test programs from the Forth
sources you need to perform the steps below. For Windows
systems, there are batch files (scripts) that simplify the
process. If you are using a Unix system, you can use these
as the basis of your own scripts.
@vskip 0pt

@subsection Windows script


@vskip -4pt
@*
In the root of the C2Forth system, you will find a batch
file called @emph{InstallWin.bat}. Run this from a console
prompt at the root of the C2Forth directory.
@vskip 0pt
@example
C:\Products\C2ForthKit.120>InstallWin @end example
@*
The script creates apermanent environment variable called
C2FORTH that specifies the root of the C2Forth directory.
If you are using Windows XP, you may have to download
@emph{setx.exe} from the Microsoft web site as it is required
by @emph{InstallWin.bat}.
@vskip 0pt
@*
The @strong{tests} directory contains @emph{banner.bld} which
can be compiled directly by VFX Forth for Windows.
@vskip 0pt

@subsection Run VFX Forth


@vskip -4pt
@*
You need any version of VFX Forth.
@vskip 0pt

@subsection Change to the C2Forth Folder


@vskip -4pt
@*
VFX Forth contains a command CD which behaves the same way as the
Unix/Windows CD shell command. Use it to change to the folder
into which you installed the C2F system. You can use DIR <cr> to
verify you are in the correct directory.
@vskip 0pt

@subsection Compile the C2F Reference Harness and Libraries


@vskip -4pt
@*
This distribution contains a sample implementation of the C2F
harness and some simple C style libraries. Change into the \HARNESS
folder and include the sources using:
@vskip 0pt
@example
  include c2f_vfx.bld <CR> @end example

@subsection Change to the \TESTS directory


@vskip -4pt
@*
On the default installation you can change to the \TESTS folder
after building the harness by performing:-
@vskip 0pt
@example
  CD ..\TESTS <cr> @end example

@subsection Compile your first test app


@vskip -4pt
@*
Compilation of a C build is performed in three steps.
@vskip 0pt
@*
The first is to initialise the C2F harness for a new compile.
@vskip 0pt
@example
  "C" <cr>                        Initialise C2F Harness @end example
@*
Second, include the source files which make up your project:
@vskip 0pt
@example
  #include banner.cxx <cr>        Include banner source @end example
@*
Third, signal the end of the build and switch back to Forth.
@vskip 0pt
@example
  "FORTH" <cr>                    Back to Forth Mode @end example
@*
If all has gone well you should see a report on the compilation.
If you perform @strong{@code{WORDS <cr>}} on the Forth system you will find that
all the C procedures are available. Type _MAIN <CR> to run the
program.
@vskip 0pt

@section Rebuilding the C toolchain


@vskip -6pt
@*
If you are going to rebuild the compiler using the same
tools as MPE, you will need a copy of Visual C++ 6.0. The
batch file @emph{setpath.bat}, adds the VC++ include directory
to the Windows search order.
@vskip 0pt
@*
If you convert the code to use other compilers, please return
your updates and scripts to MPE for the benefit of others.
@vskip 0pt
